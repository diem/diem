\section{Correctness} \label{correctness}

It is useful to think about correctness in the following manner. Each round in the chain represents a slot in a sequence of consensus decisions.
Once a decision is committed, it takes effect immediately in the next slot. For normal transactions, this means that the state of the ledger is
updated by each slot, starting with the state of the previous slot. For control actions like reconfiguration, it changes the algorithm itself.
This again, takes effect in the immediate slot following the control transaction.
Generally, note that reconfiguration transactions can change any aspect of the algorithm, not just the validator set.

Now, imagine that an entire consensus algorithm is done separately for each round/slot. Clearly, to agree on the commit output of the algorithm
we must have agreement on the algorithm itself. Initially, the system is bootstrapped with an algorithm (and in particular, the
validator-configuration) responsible for slot 1 pre-determined and known to all via genesis transaction. Thereafter, if a slot commits an algorithm (configuration) change,
it takes effect in the succeeding slot.

It is fairly obvious to see inductively that in this manner, we maintain agreement on the algorithm for each slot.

\paragraph{Enters pipelining}

As depicted above under “Pipelined reconfiguration”, the LibraBFT consensus algorithm “spreads” the phases of the protocol (for every slot) over
3 rounds. More specifically, every phase is carried in a single round and contains a new proposal. For example, the leader of round k drives only
a single phase of certification of its proposal. In the next round, k+1, a leader again drives a single phase of certification. Interestingly,
this phase has multiple purposes:
\begin{itemize}
\item The k+1 leader sends its own k+1 proposal.
\item It also piggybacks the QC for the k proposal. In this way, certifying at round k+1 generates a QC for k+1, but also a QC-of-QC for k.
\item In the third round, k+2, the k proposal can become committed, the k+1 proposal can obtain a QC-of-QC, and the k+2 can obtain a QC.
\end{itemize}

Importantly, it should be understood that spreading phases of the k-protocol into rounds k+1 and k+2 does not shift the responsibility away from the k-algorithm.

To complicate matters, if any phase aborts due to a timeout, the k-protocol remains undecided. It can become committed only via a transaction that extends the k branch. If the k command is a reconfiguration command, the reconfiguration takes effect only upon the next commit.

For example, say that round k+1 aborts (no QC(k+1) obtained). The leader for k+2 extends k, and if three consecutive rounds (k+2, k+3, k+4) complete, then the prefix of the branch up to and include k+2 becomes committed, including k. Importantly, in this case, the reconfiguration transaction takes effect at round k+2.

A correctness “meta argument” reduction is the follows:

If the algorithm for a slot k is A, then it is necessary and sufficient for the quorums used for three slots succeeding the first commit on a branch extending A to contain quorums of A. If there is no gap, then rounds k, k+1 and k+2 use algorithm A. If there is a gap and k’ is the next commit, then rounds from k all the way to k’, k’+1, k’+2 use algorithm A.

If round k* (once committed) changes the configuration to A’, then similarly, it is necessary and sufficient for slots succeeding k* to contain quorums for A’ but also A until k* becomes committed.

\paragraph{Reconfiguration condition with pipelining}
Consider a round-k proposal such that:
\begin{itemize}
\item the last committed reconfiguration transaction on the branch which the k proposal extends is A (hence, A is “in effect”)
\item the k proposal changes (once committed) the algorithm from A to A’
\item k becomes committed by a round k* transaction which extends the k-branch
\end{itemize}
Then the following is necessary and sufficient for reconfiguration to not fork:
\begin{itemize}
\item A’ should take effect at round k+1
\item rounds k+1 thru k*+2 must use QC’s that contains quorums of both A and A’.
\end{itemize}

This “meta argument” suffices to show that the above proposals all work. The proposals differ in other properties they provide, such as ease of implementation, building state-transfer transition into reconfiguration, etc.
